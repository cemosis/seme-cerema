
 %\documentclass{beamer}
%\newtheorem{Theoreme}{Theorem}
%\mode<presentation>
%{
 % \usetheme{Warsaw}
  % or ...

  %\setbeamercovered{transparent}
  % or whatever (possibly just delete it)
%}
%\usepackage[english]{babel}
% or whatever

%\usepackage[utf8]{inputenc}
% or whatever

%\usepackage{times}
%\usepackage[T1]{fontenc}
% Or whatever. Note that the encoding and the font should match. If T1
% does not look nice, try deleting the line with the fontenc.
%\title[Utilisation de l'équation de transport pour la prévision du bruit Vers un nouveau modèle %d'acustique] % (optional, use only with long paper titles)
%{Utilisation de l'équation de transport pour la prévision du bruit Vers un nouveau modèle %d'acustique}

%\author[Groupe 1] 
%{~ Groupe 1 \inst{1}}
% - Give the names in the same order as the appear in the paper.
% - Use the \inst{?} command only if the authors have different
%   affiliation.
%\institute[Université de Strasbourg] % (optional, but mostly needed)
%{
 % Université de Strasbourg
  %}
% - Use the \inst command only if there are several affiliations.
% - Keep it simple, no one is interested in your street address.
%\date[\today] % (optional, should be abbreviation of conference name)
%{\today}
% - Either use conference name or its abbreviation.
% - Not really informative to the audience, more for people (including
%   yourself) who are reading the slides online
%\subject{ddd}
% This is only inserted into the PDF information catalog. Can be left
% If you have a file called "university-logo-filename.xxx", where xxx
% is\section{Le modèle proposé}
%\subsection{The Basic Problem That We Studied}

\scriptsize

\begin{frame}{Le modèle proposé}
  % - A title should summarize the slide in an understandable fashion
  %   for anyone how does not follow everything on the slide itself.
\begin{itemize}
  %\item
   % Use \texttt{itemize} a lot.
%\Large
\item Le modèle proposé repose sur une equation de transport et a pour objectif de modéliser le champ sonored'un local de volume $V$ et  de surface de ses parois $S$. 
\item
 Le système d'équations proposé est composé respectivement d'une équation volumique modélisant   les phénomès acoustiques au sein du vlume  de la salle d'étude:
\begin{eqnarray*}
%\begin{cases}
%\begin{split}
\frac{\partial w(\vec{r},\theta,\phi,t)}{\partial t}&=& -	\vec{v} \cdot\vec{\nabla}w(\vec{r},\theta,\phi,t))-M\vec{v} w(\vec{r},\theta,\phi,t) \\
&+&  w_{sce} (\vec{r},\theta,\phi,t) \ \ \  \vec{r}\in dV,
\label{CHeq}
%\end{cases}
%\end{split}
\end{eqnarray*}
\end{itemize}
La grandeur $w$ correspond à la densité d'énergie acoustique est en $W m^{-3}s$. Le terme $\vec{v}$ est la vitesse de propagation des particules (vitesse du son), $(\theta,\phi)$  donnant la direction de propagation. Le terme $M$ est un coefficient d'atténuation atmosphérique $(0\leq M \leq 1)$
\end{frame}
\begin{frame}{Conditions aux limites} 
\begin{itemize}
\item Une première condition aux limites correspond finalement(voir \cite{Foy} ) à  l'expression de l'énergie absorbée au niveau de la paroi($S$), avec $\vec{n}$ normale extérieure. 
\end{itemize}
\begin{equation*}
\begin{cases}
\begin{split}
d_t E_{abs}(\vec{r},\theta,\phi,t)&=& (\vec{v}\cdot \vec{n}) w (\vec{r},\hat{\theta},\hat{\phi},t) 
= \alpha(1-d) (\hat{\vec{v}} \cdot \vec{n}) w (\vec{r},\hat{\theta},\hat{\phi},t)  \\
&+& \frac{\alpha d}{\pi} \int_{0}^{2\pi}\int_{0}^{\frac{\pi}{2}} (\frac{ \vec{v}}{\left | v \right |} \cdot \vec{n}) (\vec{v}' \cdot \vec{n})  w(\vec{r},\theta',\phi',t) \sin(\theta') \, d\theta' \,d\phi'    
\end{split}
\end{cases}
\end{equation*}
\begin{equation}
\mbox{si}  \ \   \vec{v} \cdot \vec{n} >0, \hat{\vec{v}} \cdot \vec{n} >0, \vec{v}' \cdot \vec{n} >0, \vec{r}\in dS 
\end{equation}
%\begin{equation*}
%d_t E_{abs}(\vec{r},\theta,\phi,t) = (\vec{v}\cdot \vec{n}) w (\vec{r},\hat{\theta},\hat{\phi},t)=0  
 %\ \ \mbox{si}\   \hat{\vec{v}} \cdot \vec{n} <0, \vec{r}\in dS
%\end{equation*}
oú
$\alpha$ coefficient d' absortion  $(0<\alpha<1)$ et 
$d$ coeffiecient de difusison dans paroi ($ 0\leq d \leq 1$)
 
\end{frame}
%\end{document}
\begin{frame}{Conditions aux limites} 
Une seconde condition aux limites correspond finalement a l'expression de l'énergie absorbée au niveau de la paroi
\begin{equation*}
\begin{cases}
\begin{split}
d_t E_{ref}(\vec{r},\theta,\phi,t) &=& (\vec{v}\cdot \vec{n}) w (\vec{r},\hat{\theta},\hat{\phi},t) 
 =R(1-d) (\hat{\vec{v}} \cdot \vec{n}) w (\vec{r},\hat{\theta},\hat{\phi},t) \\
&+&
\frac{R d}{\pi} \int_{0}^{2\pi}\int_{0}^{\frac{\pi}{2}} (\frac{ \vec{v}}{\left | v \right |} \cdot \vec{n}) (\vec{v}' \cdot \vec{n})  w(\vec{r},\theta',\phi',t) \sin(\theta') \, d\theta' \,d\phi'
\end{split}
\end{cases}
\end{equation*}
\begin{equation}
\mbox{si}  \ \   \vec{v} \cdot \vec{n} <0, \hat{\vec{v}} \cdot \vec{n} >0, \vec{v}' \cdot \vec{n} >0, \vec{r}\in dS 
\end{equation}
%\begin{equation*}
%d_t E_{abs}(\vec{r},\theta,\phi,t) = (\vec{v}\cdot \vec{n}) w (\vec{r},\hat{\theta},\hat{\phi},t)=0  
 %\ \ \mbox{si}\   \hat{\vec{v}} \cdot \vec{n} <0, \vec{r}\in dS
%\end{equation*}
oú
$R $ coefficient de réflexion  $(0<R<1)$ et 
$d$ coeffiecient de difussion dans le paroi $( 0\leq d \leq 1)$ \\
Par définition, on a : $R+ \alpha=1$.
\end{frame}
%falta solo ortografia 
\begin{frame}
Dans cette parte, on va travailler  avec le suivant probléme avec condtion aux limites
\begin{eqnarray*}
\frac{\partial w(\vec{r},\theta,\phi,t)}{\partial t}&=& -	\vec{v} \cdot\vec{\nabla}w(\vec{r},\theta,\phi,t))-M\vec{v} w(\vec{r},\theta,\phi,t) \\
&+&  w_{sce} (\vec{r},\theta,\phi,t) \ \ \  \vec{r}\in dV,
\end{eqnarray*}
\begin{equation*}
\begin{cases}
\begin{split}
d_t E_{ref}(\vec{r},\theta,\phi,t) &=& (\vec{v}\cdot \vec{n}) w (\vec{r},\hat{\theta},\hat{\phi},t) 
 =R(1-d) (\hat{\vec{v}} \cdot \vec{n}) w (\vec{r},\hat{\theta},\hat{\phi},t) \\
&+&
\frac{R d}{\pi} \int_{0}^{2\pi}\int_{0}^{\frac{\pi}{2}} (\frac{ \vec{v}}{\left | v \right |} \cdot \vec{n}) (\vec{v}' \cdot \vec{n})  w(\vec{r},\theta',\phi',t) \sin(\theta') \, d\theta' \,d\phi'
\end{split}
\end{cases}
\end{equation*}
\begin{equation}
\mbox{si}  \ \   \vec{v} \cdot \vec{n} <0, \hat{\vec{v}} \cdot \vec{n} >0, \vec{v}' \cdot \vec{n} >0, \vec{r}\in dS 
\end{equation}
\end{frame}
%\end{document}
\begin{frame}{Résultats connus pour un couloir}
Dans \cite{Jin} Jing considére le suivant modéle tridimensionelle avec conditions aux limites 
\begin{equation}
\begin{cases}
\begin{split}
&& \frac{1}{c} \frac{\partial \psi}{\partial t} (r, \Omega,t) + \Omega \cdot \nabla \psi(r, \Omega,t) + M \psi (r, \Omega,t) = \frac{Q(r,t)}{4\pi}  \\
&&0< x< L \ \ \ (y,z) \in A \\
&& \psi(r, \Omega,t) = R \left( (1-s) \psi(r,\hat{\Omega},t) + \frac{s}{\pi} \int_{\Omega' \cdot n>0} (\Omega' \cdot n) \psi(r,\Omega',t) \, d\Omega' \right), \\
&&0<x<L (y,z) \in \partial A, \Omega\cdot n < 0
\end{split}
\end{cases}
\end{equation}
\end{frame}
%\end{document}
%% \appendix
%% \section<presentation>*{\appendixname}
%% \subsection<presentation>*{Bibliographie}
%% \begin{frame}[allowframebreaks]
%%   \frametitle<presentation>{Bibliographie}
%%   \begin{thebibliography}{10}
%%   \beamertemplatearticlebibitems
%%   % Followed by interesting articles. Keep the list short. 
%% \bibitem{Jin}
%%   Yun jing 
%%     \newblock {\em One-dimensional transport equation models for sound energy propagation in long spaces: Theory}
%%   \bibitem{Foy}
%%   Cédric Foy
%%     \newblock {\em Simulation de l’acoustique intérieure d’un bâtiment par la résolution numérique d’une équation de diffusion : introduction de la diffusivité aux parois}
%%   \end{thebibliography}
%% \end{frame}
%%\end{document}
